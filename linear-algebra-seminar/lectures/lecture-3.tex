\section{Лекция 10.01.2024}

\subsection{Сумма и пересечение пространств}

\subsubsection{Способы задания подпространств в $F^n$}

\begin{enumerate}
  \item Как линейная оболочка векторов
  \item Как множество решений некоторой ОСЛУ
\end{enumerate}

\begin{comment}~

    Переход между ними сводится к поиску ФСР
\end{comment}

\begin{lemma}

  \begin{enumerate}
    \item Если $U = <u_1, \dots, u_k>$ и $W = <w_1, \dots, w_n>$, то $U + W = <u_1, \dots, u_k, w_1, \dots, w_n$
    \item Если $U: Ax = 0$ и $W: Bx = 0$, то $U \cap W:
      \begin{pmatrix}
        A \\ B
      \end{pmatrix} x = 0$
  \end{enumerate}

\end{lemma}

\bigskip

\begin{problem}~

  Для заданных (I или II способом)

  $U, W \in F^n$. Найти базис $U + W$ и базис $U \cap W$

  \begin{enumerate}
    \item Если $U$ и $W$ заданы первым способом, то легко найти базис $U + W$
    \item Если $U$ и $W$ заданы вторым способом, то легко найти базис $U \cap W$
    \item В общем случае можно перейти к удобному способу задания
  \end{enumerate}

\end{problem}

\bigskip

\begin{problem}~

  Найти базис суммы и базис пересечения для:

  $U = <
  \begin{pmatrix}
    1 \\ 2 \\ 0 \\ 1
  \end{pmatrix},
  \begin{pmatrix}
    1 \\ 1 \\ 1 \\ 0
  \end{pmatrix}, \; W = <
  \begin{pmatrix}
    1 \\ 0 \\ 1 \\ 0
  \end{pmatrix},
  \begin{pmatrix}
    1 \\ 3 \\ 0 \\ 1
  \end{pmatrix}>$

  \begin{solution}

    \begin{enumerate}
      \item Записываем в систему по столбцам и пользуемся элементарными преобразованиями строкам, что дает сохранение линейных зависимостей
        $$U + W = <
        \begin{pmatrix}
          1 \\ 2 \\ 0 \\ 1
        \end{pmatrix},
        \begin{pmatrix}
          1 \\ 1 \\ 1 \\ 0
        \end{pmatrix},
        \begin{pmatrix}
          1 \\ 0 \\ 1 \\ 0
        \end{pmatrix}>$$
      \item Записываем в систему по строкам и пользуемся элементарными преобразованиями строк, что дает сохранение линейной оболочки
      \item Записываем векторы двух систем по строкам
        $$U \cap W:
        \begin{pmatrix}
          -2 & 1 & 1 & 0 \\
          1 & -1 & 0 & 1 \\
          -3 & 1 & 3 & 0 \\
          0 & -1 & 0 & 3
        \end{pmatrix} \leadsto
        \begin{pmatrix}
          1 & 0 & 0 & -2 \\
          0 & 1 & 0 & -3 \\
          0 & 0 & 1 &
        \end{pmatrix}$$
      \item Базис $U \cap W:
        \begin{pmatrix}
          2 & 3 & 1 & 1
        \end{pmatrix}$
    \end{enumerate}

  \end{solution}

\end{problem}

\bigskip

\begin{theorem}~

  $U, W \leq V$ - подпространства, то $\dim(U + W) = \dim U + \dim W - \dim(U \cap W)$

  \begin{comment}~
    В частности, если из этих 4-ех чисел известо 3, то и 4-ое можно легко найти.

    Пэтому, если $U, W \in F^n$ заданы одним и тем же способом, то задача вычислить $\dim(U + W)$ и $\dim(U \cap W)$ решается легко
  \end{comment}

\end{theorem}

\bigskip

\begin{lemma} Алгоритм:

  Альтернативный способ построения базиса $U \cap W$ для $U, W$ заданных 1 способом

  Пусть $U = <u_1, \dots, u_k>, W = <w_1, \dots, w_l>$

  \begin{enumerate}
    \item $\forall v \in U \cap W: \exists \alpha_i, \mu_j: v = \dots$
    \item Это ОСЛУ на коэффиценты $\alpha_i, \mu_j$. $\lambda_1 \cdot u_1 + \dots + \lambda_k \cdot u_k - \mu_1 \cdot w_1 - \dots - \mu_l \cdot w_l = 0$
  \end{enumerate}

  \begin{enumerate}
    \item Записать векторы $u_1, \dots, u_k$ в столбцы матрицы $B$
    \item Привести матрицу $(A|B)$ к улучшенному ступенчатому виду (целиком)
    \item По улучшенному ступенчатому виду выписать ФСР для нашей ОСЛУ (на самом деле мы ищем ФСР для матрицы $(A|-B)$)
    \item Для каждого элемента $(\lambda_1, \dots, \lambda_k, \mu_1, \dots, \mu_l)$ из ФСР находим вектор $v$ по одной из формул: $v = \lambda_1 \cdot u_1 + \dots + \lambda_k \cdot u_k = \mu_1 \cdot w_1 + \dots + \mu_l \cdot w_l$
    \item Полученный набор векторов $v$ порождают $U \cap W$, остается выделить базис их линейной оболочки
  \end{enumerate}

  \begin{comment}
    Если $U, W$ заданы первым способом, то для нахождения базиса суммы нужно привести ту же матрицу $(A|B)$ к ступенчатому виду
  \end{comment}

\end{lemma}

\bigskip

\begin{problem}~

  Альтернативный способ для $U$ и $W$ из задачи 1

  \begin{solution}
    \begin{enumerate}
      \item $A =
        \begin{pmatrix}
          1 & 1 \\
          2 & 1 \\
          0 & 1 \\
          1 & 0
        \end{pmatrix}, \; B =
        \begin{pmatrix}
          1 & 1 \\
          0 & 3 \\
          1 & 0 \\
          0 & 1
        \end{pmatrix}$
      \item $(A|B) \leadsto
        \begin{pmatrix}
          1 & 1 & 1 & 1 \\
          0 & -1 & 2 & 1 \\
          0 & 0 & -1 & 1 \\
        \end{pmatrix} \leadsto
        \begin{pmatrix}
          1 & 0 & 0 & 1 \\
          0 & 1 & 0 & 7 \\
          0 & 0 & 1 & -1 \\
        \end{pmatrix}$
      \item ФСР: $
        \begin{pmatrix}
          1 & 1 & 1 & 1
        \end{pmatrix}$
      \item $v = 1 \cdot
        \begin{pmatrix}
          1 \\ 2 \\ 0 \\ 1
        \end{pmatrix} + 1 \cdot
        \begin{pmatrix}
          1 \\ 1 \\ 1 \\ 0
        \end{pmatrix} =
        \begin{pmatrix}
          2 \\ 3 \\ 1 \\ 1
        \end{pmatrix} = 1 \cdot
        \begin{pmatrix}
          1 \\ 0 \\ 1 \\ 0
        \end{pmatrix} + 1 \cdot
        \begin{pmatrix}
          1 \\ 3 \\ 0 \\ 1
        \end{pmatrix}$
      \item Базис $U \cap W: \left(
          \begin{pmatrix}
            2 \\ 3 \\ 1 \\ 1
        \end{pmatrix}\right)$
    \end{enumerate}
  \end{solution}

\end{problem}

\bigskip

\begin{lemma}~

  Если в альтернативном алгоритме каждый из 2-х наборов $u_1, \dots, u_k$ и $w_1, \dots, w_l$ является базисом, то полученные на шаге 4 векторы $v$ автоматически образуют вектор пересечения (5 шаг не нужен)

  \begin{proof}
    \begin{enumerate}
      \item Пусть $s$ - количество векторов в ФСР, полученной на шаге 3, тогда $s = (k + l) - rk(A|B) = \dim U + \dim W - \dim(U + W) = \dim(U \cap W)$
      \item
    \end{enumerate}
  \end{proof}

  \begin{comment}
    Формула включения-исключения не обощается для размерностей множеств. Для $n = 3$ можно построить контр-пример

    Проблема возникает из-за $U_1 \cap (U_2 + U_3) \not= U_1 \cap U_2 + U_1 \cap U_3$
  \end{comment}

\end{lemma}

\bigskip

\begin{problem}~

  Пусть $A \in M_n(F)$, $\overline{A}$ - присоединенная матрица

  \begin{enumerate}
    \item $rkA = n \Rightarrow rk\overline{A}$ (так как она обратима)
    \item $rkA \leq n - 2 \Rightarrow \overline{A} = 0 \Rightarrow rk \overline{A}$
    \item $rk = n - 1 \Rightarrow rk \overline{A} = 1$
  \end{enumerate}

\end{problem}
