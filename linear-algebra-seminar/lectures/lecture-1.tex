\section{Лекция 5.12.2024}


\begin{problem}

    Пусть $F$ - поле, $v_1, \dots v_m \in F^n$, которые являются линейно независимыми. Дополнить до базиса $F^n$.

    \begin{solution}~
        
        \begin{enumerate}
            \item В матрицу $A \in Mat_{n \times (m + n)}(F)$ по столбцам запишем векторы $v_1, \dots v_m, e_1, \dots, e_n$
            \item Элементарными строк приводим $A$ к ступенчатому виду $A'$
            \item В качестве базиса $F^n$ возьмем те векторы из набора $v_1, \dots v_m, e_1, \dots, e_n$, номера которых совпадают с номерами столбцов, с лидерами строк (среди них будут $v_1, \dots v_m$)
        \end{enumerate}

    \end{solution}

    \bigskip

    \begin{example}~
        
        \begin{enumerate}
            \item $v_1 = \begin{pmatrix}
                1 \\ 2 \\ 3 \\ 4
            \end{pmatrix}, \;\; v_2 = \begin{pmatrix}
                2 \\ 4 \\ 5 \\ 6
            \end{pmatrix}$
            $$A = \begin{pmatrix}
                1 & 2 & 1 & 0 & 0 & 0 \\
                2 & 4 & 0 & 1 & 0 & 0 \\
                3 & 5 & 0 & 0 & 1 & 0 \\
                4 & 6 & 0 & 0 & 0 & 1
            \end{pmatrix} \leadsto \begin{pmatrix}
                1 & 2 & 1 & 0 & 0 & 0 \\
                0 & -1 & -3 & 0 & 1 & 0 \\
                0 & 0 & -2 & 1 & 0 & 0 \\
                0 & 0 & 0 & 1 & -2 & 1
            \end{pmatrix}$$
            \item Берем первые 4 столбца (с лидерами строк) - $v_1, v_2, e_1, e_2$
        \end{enumerate}

    \end{example}

    \bigskip

    \begin{solution}~

        \begin{enumerate}
            \item \textit{Рассмотрим более эффективный способ}
            \item В матрицу $A$ запишем $v_1, \dots, v_m$
            \item Элементарными преобразованиями столбцов приведем к транспонированному ступенчатому виду ($A'$)
            $$\begin{pmatrix}
                * & 0 & 0 & \dots \\
                * & * & 0 & \dots \\
                \vdots & \vdots & \vdots & \ddots
            \end{pmatrix}$$
            \item Дополнить набор столбцов $A'$ до базиса $F^n$ теми векторами стандартного базиса, номера которых не являются номерами ведущих элементов столбцов
            \item Эти векторы дополняют и $v_1, \dots, v_m$ до базиса
        \end{enumerate}
        
    \end{solution}

    \bigskip

    \begin{example}~

        \begin{enumerate}
            \item $v_1 = \begin{pmatrix}
                1 \\ 2 \\ 3 \\ 4
            \end{pmatrix}, \;\; v_2 = \begin{pmatrix}
                2 \\ 4 \\ 5 \\ 6
            \end{pmatrix}$
            $$A = \begin{pmatrix}
                1 & 2 \\
                2 & 4 \\
                3 & 5 \\
                4 & 6
            \end{pmatrix} \leadsto \begin{pmatrix}
                1 & 0 \\
                2 & 0 \\
                3 & -1 \\
                4 & -2
            \end{pmatrix}$$
            \item Нет ведущего элемента в 2 и 4 строках, а значит этими векторами и надо дополнить нашу систему: $v_1, v_2, e_2, e_4$
        \end{enumerate}
        
    \end{example}

    \bigskip

    \begin{comment}~
        Дополнение до базиса не единственное. Вместо любого дополняющего набора можно взять любые векторы из его линейной оболочки
    \end{comment}

\end{problem}


\bigskip


\subsection{Ранг матрицы}

\begin{definition} Ранг набора
    
    Пусть $V$ - векторное подпространство над $F$ и $v_1, \dots, v_m \in V$.

    \textit{Рангом набора векторов} называется $\dim<v_1, \dots, v_m$

    Ранг сохраняется приэлементарных преобразованиях

\end{definition}


\bigskip


\begin{definition} Ранг матрицы
    
    Пусть $A \in Mat_{m \times n}(F)$

    \textit{Рангом матрицы $A$} называется ранг ее системы строк (как векторов в $F^n$). 

    Обозначается $rk A$

    \begin{properties}~

        \begin{enumerate}
            \item $rk A = rk A^T$
            \item $0 \leq rk A \leq \min\{m, n\}$
            \item $rk A = 0 \Leftrightarrow A = 0$
            \item Ранг матрицы сохраняется при элементарных преобразованиях как строк, так и столбцов
            \item Ранг $A$ равен количеству ненулевых строк в ступенчатом виде
        \end{enumerate}
        
    \end{properties}

\end{definition}


\bigskip


\begin{lemma}
    
    Пусть $A \in Mat_{m \times n}(F)$

    $$rk A = 1 \Leftrightarrow A = f \cdot c^T$$

    где $b \in Mat_{m \times 1}(F), \; c \in Mat_{n \times 1}, \; b \not= 0, \; c \not= 0$

    $$\begin{pmatrix}
        b_1 \\ \vdots \\ b_m
    \end{pmatrix} \cdot \begin{pmatrix}
        c_1 & \dots & c_n
    \end{pmatrix} = \begin{pmatrix}
        b_1 \cdot c_1 & \dots & b_1 \cdot c_n \\
        \vdots & \ddots & \vdots \\
        b_m \cdot c_1 & \dots & b_m \cdot c_n
    \end{pmatrix} = \begin{pmatrix}
        b_1 \cdot c^T \\ \vdots \\ b_m \cdot c^T
    \end{pmatrix} = \begin{pmatrix}
        c_1 \cdot b & \dots & c_n \cdot b
    \end{pmatrix}$$

    \begin{proof}~

        \begin{enumerate}
            \item Пусть $b$ - ненулевой столбец $A, \; \forall j = 1, \dots n \; \exists \lambda_j \in F: \; A^{(j)} = \lambda_j \cdot b$
            \item Тогда можно взять $c = \begin{pmatrix}
                \lambda_1 \\ \vdots \\ \lambda_n
            \end{pmatrix}$ и $A = b \cdot c^T$
            \item Из разложение по столбцам обратный факт очевиден. (Берем $c \not= 0$, получаем ненулевой столбец $\dots$)
        \end{enumerate}
        
    \end{proof}

\end{lemma}


\bigskip


\begin{lemma}
    
    Пусть $V$ - векторное пространство над $F$. 
    
    $$u_1, \dots, u_n \in V$$
    $$v_1, \dots, v_m \in V$$
    
    $$\dim<u_1, \dots, u_n, v_1, \dots, v_m> \;\; \leq \; \dim<u_1, \dots, u_n> + \dim<v_1, \dots, v_m>$$

    \begin{proof}~

        \begin{enumerate}
            \item Пусть $u_{j_1}, \dots, u_{j_s}$ - базис векторной оболочки $<u_1, \dots, u_n>$
            \item $v_{k_1}, \dots, v_{k_p}$ - базис векторной оболочки $<v_1, \dots, v_m>$
            \item $r = \dim<u_1, \dots, u_n>, \; s = \dim<v_1, \dots, v_m>$
            \item $<u_1, \dots, v_m> = <u_{j_1}, \dots, v_{k_p}$
            \item $\dim< \dots > \leq r + s$
        \end{enumerate}
        
    \end{proof}

\end{lemma}


\bigskip


\begin{lemma}
    
    Пусть $A, B \in Mat_{m \times n}(F)$

    \begin{properties}~
        
        \begin{enumerate}
            \item $rk(A + B) \leq rk A + rk B$
            \item $rk(A - B) \geq rk A - rk B$
        \end{enumerate}

    \end{properties}

    \bigskip

    \begin{proof} 1~

        \begin{enumerate}
            \item $A = \left(a_1, \dots, a_n\right), B = \left(b_1, \dots, b_n\right)$
            \item $A + B = \left(a_1 + b_1, \dots, a_n + b_n\right)$
            \item $<a_1 + b_1, \dots, a_n + b_n> \leq <a_1, \dots, b_n$
            \item $rk(A + B) \leq rk A + rk B$
        \end{enumerate}
        
    \end{proof}

    \bigskip

    \begin{proof} 2~

        $rk A = rk ((A - B) + B) \leq rk (A - B) + rk B$
        
    \end{proof}

    \bigskip

    \begin{comment}
    
        Эти неравенства достигаются (например при $B = 0$)

    \end{comment}

\end{lemma}


\bigskip


\begin{lemma}
    
    Пусть $A \in Mat_{m \times n}(F), \; r = rk A$

    \begin{properties}~
        
        \begin{enumerate}
            \item $A = B_1 + \dots + B_s, \;\; rk B_i = 1 \;\; \forall i \Rightarrow s \geq r$
            \item $\exists B_1, \dots, B_r, \; rk B_i = 1: \; A = B_1 + \dots + B_r$
        \end{enumerate}

    \end{properties}

    \bigskip

    \begin{proof} 1~

        $r = rk (B_1 + \dots + B_s) \leq rk B_1 + \dots + rk B_s = s$
        
    \end{proof}

    \bigskip

    \begin{proof} 2~

        \begin{enumerate}
            \item $A = \left(a_1, \dots, a_r\right)$
            \item Пусть $a_1, \dots, a_r$ - базис $<a_1, \dots, a_r$
            \item $\forall 1 \leq k \leq n - r: \;\; a_{r + k} = \lambda_{k1} \cdot a_1 + \dots + \lambda_{kr} \cdot a_r$
            \item тогда можно взять $B_i = a_i \cdot (0 \dots 0 1 \; (\text{$i$ - ое место}) \; 0 \dots 0 \lambda_{1i} \dots \lambda_{n-r i})$
            \item Первые $r$ - элементов - нули с 1 посередине
        \end{enumerate}
        
    \end{proof}

    \bigskip

    \begin{proof} Алгоритм~
        
        \begin{enumerate}
            \item Найти максимальную линейно-независимую системы столбцов $A^{(i_1)}, \dots, A^{(i_r)}$
            \item В матрице $A$ (т. е. найти базис пространства столбцов)
            \item Линейно выразить остальные столбцы через этот базис
            \item В $j$ - ый столбец матрицы $B_k$ записатт ту компоненту разложения столбца $A^{(j)}$ по базису $A^{(i_1)}, \dots, A^{(i_r)}$, которая пропорциональна $A^{(i_k)}$ (другими словами $B_k$ это столбец $B_k = A^{(i_k)} \cdot b_k^T$)
        \end{enumerate}

    \end{proof}

    \bigskip

    \begin{example}~
        
        \begin{enumerate}
            \item $A = \begin{pmatrix}
                1 & 1 & 0 & 2 \\
                3 & -3 & 2 & 0 \\
                2 & -1 & 1 & 1
            \end{pmatrix} \leadsto \begin{pmatrix}
                1 & 0 & 1/3 & 1 \\
                0& 1 & -1/3 & 1 \\
            \end{pmatrix}$
            \item $A^{(1)}, A^{(2)}$ - базис
            $$A^{(3)} = \frac{1}{3} \cdot A^{(1)} - \frac{1}{3} \cdot A^{(2)}$$
            $$A^{(4)} = 1 \cdot A^{(1)} + 1 \cdot A^{(2)}$$
            \item Пишем разложение:
            $$B_1 = \begin{pmatrix}
                1 \\ 3 \\ 2
            \end{pmatrix} \cdot \begin{pmatrix}
                1 & 0 & 1/3 & 1
            \end{pmatrix}$$
            $$B_2 = \begin{pmatrix}
                1 \\ -3 \\ -1
            \end{pmatrix} \cdot \begin{pmatrix}
                0 & 1 & -1/3 & 1
            \end{pmatrix}$$
        \end{enumerate}

    \end{example}

\end{lemma}