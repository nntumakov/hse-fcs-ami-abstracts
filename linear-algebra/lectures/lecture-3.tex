\section{Лекция 16.12.2024}


\begin{lemma}~
    
    $V$ - векторное пространство над полем $F, \dim V < \infty$
    
    $U, W \subseteq V$ - подпространства
    
    Новые пространства:

    \begin{enumerate}
        \item $U \cap W$
        \item $U + W = \{u + w \; | \; u \in U, \; w \in W\}$
    \end{enumerate}

    \begin{comment}
        
        $$\dim(U \cap W) + \dim(U + W) = \dim U + \dim W$$

    \end{comment}

\end{lemma}


\bigskip


\begin{definition}~
    
    Пусть $U_1, \dots, U_m$ - набор подпространтсв

    \textit{Сумма подпространтсв $U_1, dots, U_m$} - это 
    
    $$U_1 + \dots + U_m = \{u_1 + \dots + u_2 \; | \; u_1 \in U_1, \dots u_m \in U_m\}$$

    \begin{comment}
        
        Сумма подпространств является подпространством.

        $$\dim(U_1 + \dots + U_m) \leq \dim U_1 + \dots + \dim U_m$$

    \end{comment}

\end{definition}


\bigskip


\begin{definition}~
    
    \textit{Подпространста $U_1, \dots, U_k$} называются линейно независимыми, если $\forall u_1 \in U_1, \dots, u_k \in U_k$ из условия $u_1 + \dots + u_k = \vec{0}$ следует $u_1 = \dots = u_k = \vec{0}$

    \begin{example}
        
        $\dim U_i = 1, \; \forall i, U_i = <e_i>$ следует $U_1, \dots, U_k$ линейно независимые, что равносильно линейной независимости $e_i$

    \end{example}

\end{definition}


\bigskip


\begin{theorem}~
    
    Слудующие условия эквиваленты:

    \begin{enumerate}
        \item $U_1, \dots, U_k$ линейно независимые
        \item $\forall u \in U_1 + \dots + U_k: \; \exists u_i \in U_i$, такие что $u = u_1 + \dots + u_k$
        \item если $e_i$ - базис в $U_i$, то $e_1 \cup \dots \cup e_k$ - базис в $U_1 + \dots + U_k$
        \item $\dim(U_1 + \dots + U_k) = \dim U_1 + \dots + \dim U_k$
        \item $\forall i: \; U_i \cap (\sqcup_{j \not= i}U_{j} = \{\vec{0}\})$ (Объединение мультимножеств: $\sqcup$)
    \end{enumerate}

    \begin{proof} $1 \rightarrow 2$

        \begin{enumerate}
            \item Пусть $u \in U_1 + \dots + U_k$ и существуют 2 представления $u = u_1 + \dots + u_k = u'_1 + \dots + u'_k$
            \item Вычтем: $(u_1 - u'_1) + \dots + (u_k - u'_k) = \vec{0} \Rightarrow u_i = u'_i$
        \end{enumerate}
        
    \end{proof}

    \begin{proof} $2 \rightarrow 3$

        \begin{enumerate}
            \item Пусть $u \in U_1 + \dots + U_k$ в силу условия 2 $u = u_1 + \dots + u_k$ (однозначно представлен)
            \item Так как $e_i$ - базис в $U_i$, то всякий $u_i$ - единственным образом представим в виде линейной комбинации векторов из $e_i$
            \item Знаем, что $u$ - однозначно представим в виде линейной комбинации векторов из $\sqcup_{i}e_i$, следовательно $\sqcup_{i}e_i$ - базис в сумме
        \end{enumerate}
        
    \end{proof}

    \begin{proof} $3 \rightarrow 4$

        $\dim(U_1 + \dots + U_m) = |\sqcup e_i| = |e_1| + \dots + |e_m| = \dim U_1 + \dots + \dim U_k$
        
    \end{proof}

    \begin{proof} $4 \rightarrow 5$

        \begin{enumerate}
            \item $\overline{U_i} = U_1 + \dots + U_{i - 1} + U_{i + 1} + U_k$
            \item $\dim(U_i \cap \overline{U_i}) = \dim U_i + \dim \overline{U_i} - \dim (U_i + \overline{U_i}) \leq \dim U_i + \dim U_1 + \dots \dim U_k - \dim U_1 - \dots - \dots \dim U_k = 0$
            \item $\dim (U_i \cap \overline{U_i}) \leq 0$
        \end{enumerate}
        
    \end{proof}

    \begin{proof} $5 \rightarrow 1$

        \begin{enumerate}
            \item Пусть $u_1 \in U_1, \dots, u_k \in U_k$, таковы, что $u_1 + u_k = \vec{0}$. Тогда для любого номера $u_i = - u_1 - \dots u_{i - 1} - u_{i + 1} - \dots u_k \in U_i \cap \overline{U_i} = \vec{0}$
        \end{enumerate}
        
    \end{proof}

    \begin{corollary}
    
        Подпространста $U, W \subseteq V$ линейно независимы, что равносильно $U \cap W = \{\vec{0}\}$

    \end{corollary}

\end{theorem}


\bigskip


\begin{definition} Разложение в прямую сумму
    
    Говорят, что векторы разлагаются в прямую сумму своих подпространств $U_1, \dots, U_k$, если:

    $$U_1, \dots, U_k$$

    и обозначают: $U_1 \oplus \dots \oplus U_k$

    \begin{example}
        
        $e_1, \dots, e_n$ - базис в $V$, то 

        $V = <e_1> \oplus \dots \oplus <e_n>$

    \end{example}

    \begin{comment}
        
        \begin{enumerate}
            \item При $k = 2$
            \item $V = U_1 \oplus U_2 \Leftrightarrow \begin{cases}
                V = U_1 + U_2 \\
                U_1 \cap U_2 = \vec{0}
            \end{cases} \Leftrightarrow \begin{cases}
                \dim V = \dim U_1 + \dim U_2 \\
                U_1 \cap U_2 = \{\vec{0}\}
            \end{cases}$
            \item $V = U_1 \oplus U_2 \Rightarrow \forall v \in V: \; \exists ! u_1 \in U_1, u_2 \in U_2: \; V = U_1 + U_2$
        \end{enumerate}

    \end{comment}

\end{definition}


\bigskip


\begin{definition}
    
    В этой ситуации $u_1$ \textit{проекцией} вектора $v$ на подпространтсво $U_1$ вдоль подпространства $U_2$

\end{definition}


\subsection{Линейные отображения - в слайдах}